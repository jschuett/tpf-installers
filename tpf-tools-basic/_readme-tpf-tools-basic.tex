% Autogenerated translation of _readme-tpf-tools-basic.md by Texpad
% To stop this file being overwritten during the typeset process, please move or remove this header

\documentclass[12pt]{book}
\usepackage{graphicx}
\usepackage{fontspec}
\usepackage[utf8]{inputenc}
\usepackage[a4paper,left=.5in,right=.5in,top=.3in,bottom=0.3in]{geometry}
\setlength\parindent{0pt}
\setlength{\parskip}{\baselineskip}
\setmainfont{Helvetica Neue}
\usepackage{hyperref}
\pagestyle{plain}
\begin{document}

\chapter*{tpf-tools-basic}

\hrule
'tpf-tools-basic' is a tool for low-latency, bidirectional transmission of audio among different locations. 
Version 1.1.1 is compatible with macOS 10.14+ (only Intel).
For further information about ‘tpf-tools-basic’, including manuals, see: \href{https://networkperformance.space}{https://networkperformance.space}.

\section*{Download:}

\begin{itemize}
\item Go to: \href{https://github.com/zhdk/tpf-installers/releases/}{https://github.com/zhdk/tpf-installers/releases/}
\item Download the 'install\emph{tpf-tools}basic.pkg ' file.
\end{itemize}

\section*{Installation:}

\begin{itemize}
\item Right-click to open the 'install-tpf-tools-basic.pkg' and follow the instructions of the installer.
\end{itemize}

 	\textbf{\emph{Note:}} The 'tpf-party.app' is included in the 'tpf-tools-intermediate'.

\section*{Start the 'tpf-tools-basic':}

\begin{itemize}
\item After installation, start your audio session: double-click the file ’tpf-party.app’ from the folder ’tpf-tools-basic’.  
\item Click on ‘Settings’ and choose your connection parameters by entering server name, room name and your name in the respective boxes. 
\item Close the ’Settings’ window. Click on the upper left box beside your name; it turns blue and the remote peers, connected to the same server and room, will appear. 
\item To connect, click on the box besides the peer name; it turns blue and it starts to blink in yellow at your remote partner’s end. By clicking on it, a connection is established.
\end{itemize}

\section*{Authors:}

\begin{itemize}
\item Roman Haefeli \href{mailto:roman.haefeli@zhdk.ch}{roman.haefeli@zhdk.ch}
\item Johannes Schütt \href{mailto:johannes.schuett@zhdk.ch}{johannes.schuett@zhdk.ch}
\item TPF-Team @ Zurich University of the Arts (ZHdK)
\end{itemize}

\section*{License}

GPL 3.0 (view LICENSE.txt)

\end{document}
